\newpage

\begin{center}
{\bf \large 关于本书}
\end{center}

为什么本书和其它的并行编程书籍不同呢?
原因在于我们主要关注在实现层面:

\begin{itemize}

\item 这里几乎没有理论内容,诸如 O() 分析、最大理论加速、PRAM、有向无环图(DAG)等等。

\item 书中使用的都是真实代码。

\item 我们使用的都是主流的并行平台,包括 OpenMP、CUDA 和 MPI,而没有使用其它
仍处于实验阶段的语言。

\item 关于性能的主题——通信延迟、内存/网络连接、负载均衡等,在全书中
交叉进行,并且都是在特定平台或应用的层面进行讨论的。

\item 相当关注 debug 技术。

\end{itemize}

书中使用的主要编程语言 C/C++,但也使用了一些 R 代码。R 已经是最主流
的用于数据分析的语言。作为一门脚本语言,R 可以用于快速原型构建。
在本书中,我用 R 将可以一些例子表述得远远比用 C/C++ 要简洁,从而
使得学生更容易的理解所使用的并行计算原则。出于同样的原因,学生们
也可以更容易地编写并行代码,更加集中精力于这些原则之上。另外,
R 也有相当丰富的并行库。

我们假设学生在编程方面是有相当经验的,并有包括线性代数在内的数学背景。
附录里回顾了本书所需要的数学知识。另一个附录提供了不同系统问题的概述,
包括进程调度和虚拟内存等。

需要特别说明的是,书中多数代码没有进行优化。我们主要关注的是技术和语言
使用的清晰明了。然而,有很多影响速度的因素值得讨论,比如高速缓存一致性问题、
网络延迟、GPU 内存结构等等。

这里展示了你可以如何使用书中的代码:
本书使用 \LaTeX 排版,原始的 {\bf .tex} 文件可以在
\url{http://heather.cs.ucdavis.edu/~matloff/158/PLN}下载。
请直接下载相关文件(文件名应该足够明),之后使用一个文本编辑器
进行裁剪,从而得到感兴趣的代码。 

为了向学生展示研究和教学的相互促进关系,我会时不时引用我的一些研究工作。

如同我的其它开源图书,本书是在\textbf{不停变动}中的。我会继续添加新的主题
、新的示例等等,当然也会修补漏洞和改善说明。由于这个原因,所以
保存本书最新版本的链接,\url{http://heather.cs.ucdavis.edu/~matloff/158/PLN/ParProcBook.pdf},
比保存一份拷贝更好。

同样出于这个原因,我非常希望得到反馈。这里我希望感谢 Stuart
Ambler、Matt Butner、Stuart Hansen、Bill Hsu、Sameer Khan、Mikel
McDaniel、Richard Minner、Lars Seeman、Marc Sosnick 和 Johan Wikstr{\"o}m 的评论。
特别感谢 Hsu 教授为我提供了高级的 GPU 设备。

各位可能对我另一本关于概览和统计的开源图书感兴趣,可以在
\url{http://heather.cs.ucdavis.edu/probstatbook}下载。

本书使用 Creative Commons Attribution-No Derivative
Works 3.0 United States License 发行。在美国境外的版权
归 Matloff 所有,在保证提供作者和发行信息的情况,这些材料
仍可用于教学使用。如果您使用了本书用于教学,我将很高兴您能
通知我,这仅仅为了让我知道这些材料正在被使用当中,但这不是必须的。
