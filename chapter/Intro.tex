% -----------------------------------------------------------------------------
% !Mode:: "TeX:UTF-8:Hard"
% -----------------------------------------------------------------------------
\chapter{并行处理入门}
\label{chap:intro}

并行设备为需要大量计算的应用程序提供了一个极佳的机会。
而对这些设备的有效使用需要对其工作原理有一个清晰的了解。
这个章节给大家提供了一个软件和硬件两方面的概述。

\section{为什么要使用并行系统?}

\subsection{执行速度}

在一些计算机用户当中,对更快的机器的需求一直在增长。
这个在苹果和皮克斯公司的创始人和 CEO,Steve Jobs 的一份回忆录中被多次强调。
他写到,当他80年代在苹果公司的时候,他总是担心别的公司
会生产出比他们更快的计算机。但后来在皮克斯的时候,
因为图形处理对高速计算的大量需求,他总是希望别人可以
生产更快的机器,那他就可以用这些机器了!

加速的一个主要来源就是操作的并行化。
并行操作既可以是处理器内部的,
比如单个处理器内部的 pipelining 或多个 ALU\footnote{Arithmetic Logic Unit,算术逻辑单元,
是处理器中的一个功能模块,用来执行诸如加减乘除以及寄存器中的值之间的逻辑运算},
也可以是处理器之间的,不同的处理器并行的处理一个问题
的不同部分。我们这里着重于处理器之间的操作。

例如,加州大学戴维斯分校的注册办公室使用
内存共享的多处理器系统来处理在线注册工作。
在线注册需要大量的数据库计算。
为了在合理的时间内处理这个计算,
将程序需要的工作分块,将数据库的不同分块分配给
不同的处理器。
数据库领域对于大型内存共享系统的商业成功做出
了巨大贡献。

正如 Pixar 的例子所示,诸如计算机图形处理等计算密集型的应用
对快速的并行计算也有相当的需求。
没人愿意为生成一张图片就等几个小时,
而并行计算可以显著地提升速度。
这里我们以 \textbf{ray tracing} 操作为例。
我们的代码需要在考虑光在不同物体上的反射和吸收的情况下,
跟踪一个场景里光线。
假设这个图像有一个行像素点组成,每一行也由1000个像素点构成。
比如说要用25个处理器并行地解决这个问题,
我们可以将图像分为200$\times$200的25块,
而由每个处理器处理其中一块。

但需要强调的是,实际情况比这有挑战得多。
首先,计算需要各个处理器直接的通信,如果不仔细处理,这会对性能有很大影响。
其次,如果真的需要很好的加速,可能需要考虑一个问题,
那就是其中一些分块比其它需要更多的计算量。下面我们还会讨论这个问题。

我们正处于大数据时代,这需要大型的计算,
因此又产生了对并行处理的一个重要需求。


\subsection{内存}

的确,当并行处理出现时,执行速度可能是多数首先想到的。
但在很多应用中,一个同样重要的考虑就是内存容量。
并行应该经常需要大量的内存,并且在很多情况下会超出
单机的极限。
如果我们有很多机器同时工作,
特别是后面提到的信息传递系统中,
我们可以解决大量的内存需求。

\subsection{分布式处理}

前面两个小节里,我们讨论了计算机科学里最重要的
两个问题——时间(速度)和空间(内存容量)。
但还有第三个原因驱使我们进行并行计算——{\bf 分布式处理}。
例如,在一个分布式数据库中,
数据库的各个部分可以在物理位置上相隔很远。
如果一个位置上的事务(transaction)处理需求上升,
我们将会通过网络来提升效率。

\subsection{我们的着重点}

在本书中,我们注意着重在处理速度上。


\section{并行处理硬件}

有一个经常出现的情景:
有人有了一个很新的并行设备,很兴奋地写了一个程序在上面运行,
但最后却发现,并行的代码比之前的串行版本还要慢!
这主要源于,至少在高层次上,对硬件工作原理缺乏了解。

这不是本针对硬件的书,但由于使用并行计算的目的在于速度,
代码的效率就是主要问题了。
这也意味着我们要对我们所使用的硬件有一个相对的了解。
在本节,我们会给给出并行硬件的一个概述。

\subsection{内存共享系统}

\subsubsection{基本构架}

这种情况下,多个 CPU 使用相同的物理内存。
这种构架一般被称作 MIMD,
全称为 Multiple Instruction
(由于不同的 CPU 独立工作,所以会执行不同的指令),Multiple Data
(在给定时间点,不同的 CPU 一般会使用不同的内存位置)。

不久以前,内存共享系统造价非常高,
一般只有诸如保险和银行这种大公司才能负担起。
高端的机器的确依然很贵,
但现在{\bf 多核}机器,两个或更多的 CPU 共享内存,在家里甚至在手机上
都是很常见的\footnote{这里的术语有一些令人困惑。
尽管每个核都是一个完整的处理器,但这个领域的人们倾向于将整个芯片成为
一个``处理器''。本书中,处理器一般会包括多核,例如,一个四核的芯片被认为有两个处理器。}!

\subsubsection{多处理器结构}

一个对称多处理器系统(symmetric multiprocessor system,SMP)包涵以下结构:

\includegraphics{Images/UMABus.jpg} 

多核系统和 SMP 的效率一样高,除了处理器都是在一个块链接在总线的芯片上。

所谓的 NUMA 架构会在第 \ref{chap:shared} 章讨论。

\subsubsection{内存问题}

我们考虑上面的 SMP 结构图。

\begin{itemize}

\item The Ps are processors, e.g. off-the-shelf chips such as Pentiums.

\item The Ms are \textbf{memory modules}. These are physically separate
objects, e.g. separate boards of memory chips.  It is typical that there
will be the same number of memory modules as processors.  In the
shared-memory case, the memory modules collectively form the entire
shared address space, but with the addresses being assigned to the
memory modules in one of two ways:

\begin{itemize}

\item (a)

High-order interleaving. Here consecutive addresses are in the
\underbar{same} M (except at boundaries). For example, suppose for
simplicity that our memory consists of addresses 0 through 1023, and
that there are four Ms.  Then M0 would contain addresses 0-255, M1 would
have 256-511, M2 would have 512-767, and M3 would have
768-1023.

We need 10 bits for addresses (since $1024 = 2^{10}$).  The two
most-significant bits would be used to select the module number (since
$4 = 2^2$); hence the term {\it high-order} in the name of this design.
The remaining eight bits are used to select the word within a module.

\item (b)

Low-order interleaving. Here consecutive addresses are in consecutive
memory modules (except when we get to the right end). In the example
above, if we used low-order interleaving, then address 0 would be in M0,
1 would be in M1, 2 would be in M2, 3 would be in M3, 4 would be back in
M0, 5 in M1, and so on.

Here the two least-significant bits are used to determine the module
number.

\end{itemize}

\item To make sure only one processor uses the bus at a time, standard bus
arbitration signals and/or arbitration devices are used.

\item There may also be \textbf{coherent caches}, which we will discuss later.

\end{itemize}

All of the above issues can have major on the speed of our program, as
will be seen later.

\subsection{信息传递系统}

\subsubsection{基础构架}

Here we have a number of independent CPUs, each with its own independent
memory.  The various processors communicate with each other via networks
of some kind.

\subsubsection{示例:集群(cluster)}

Here one has a set of commodity PCs and networks them for use as a
parallel processing system. The PCs are of course individual machines,
capable of the usual uniprocessor (or now multiprocessor) applications,
but by networking them together and using parallel-processing software
environments, we can form very powerful parallel systems.

One factor which can be key to the success of a cluster is the use of a fast
network, fast both in terms of hardware and network protocol.  Ordinary
Ethernet and TCP/IP are fine for the applications envisioned by the
original designers of the Internet, e.g. e-mail and file transfer, but
is slow in the cluster context.  A good network for a cluster is, for instance,
Infiniband.

Clusters have become so popular that there are now ``recipes'' on how to
build them for the specific purpose of parallel processing.  The term
{\bf Beowulf} come to mean a cluster of PCs, usually with a fast network
connecting them, used for parallel processing.  Software packages such
as ROCKS (\url{http://www.rocksclusters.org/wordpress/}) have been
developed to make it easy to set up and administer such systems.

\subsection{SIMD}

In contrast to MIMD systems, processors in SIMD---Single Instruction,
Multiple Data---systems execute in lockstep.  At any given time, all
processors are executing the same machine instruction on different data.

Some famous SIMD systems in computer history include the ILLIAC and
Thinking Machines Corporation's CM-1 and CM-2.  Also, DSP (``digital
signal processing'') chips tend to have an SIMD architecture.

But today the most prominent example of SIMD is that of GPUs---graphics
processing units.  In addition to powering your PC's video cards, GPUs
can now be used for general-purpose computation.  The architecture is
fundamentally shared-memory, but the individual processors do execute in
lockstep, SIMD-fashion.

\section{Programmer World Views}

\subsection{示例:矩阵向量相乘}
\label{matvec}

To explain the paradigms, we will use the term \textbf{nodes}, where
roughly speaking one node corresponds to one processor, and use the
following example:

\begin{quote}

Suppose we wish to multiply an nx1 vector X by an nxn matrix A, putting
the product in an nx1 vector Y, and we have p processors to share the
work.

\end{quote}

In all the forms of parallelism, each node could be assigned some of the
rows of A, and that node would multiply X by those rows, thus forming
part of Y.

Note that in typical applications, the matrix A would be very large, say
thousands of rows, possibly even millions.  Otherwise the computation
could be done quite satisfactorily in a {\bf serial}, i.e.
nonparallel manner, making parallel processing unnecessary..

\subsection{内存共享}

\subsubsection{Programmer View}
\label{sharedview}

In implementing the matrix-vector multiply example of Section
\ref{matvec} in the shared-memory paradigm, the arrays for A, X and Y
would be held in common by all nodes. If for instance node 2 were to
execute

\begin{Verbatim}
   Y[3] = 12;
\end{Verbatim}

and then node 15 were to subsequently execute

\begin{Verbatim}
   print("%d\n",Y[3]);
\end{Verbatim}

then the outputted value from the latter would be 12.

Computation of the matrix-vector product AX would then involve the nodes
somehow deciding which nodes will handle which rows of A.  Each node
would then multiply its assigned rows of A times X, and place the result
directly in the proper section of the shared Y.

Today, programming on shared-memory multiprocessors is typically done
via \textbf{threading}.  (Or, as we will see in other chapters, by
higher-level code that runs threads underneath.) A \textbf{thread} is
similar to a \textbf{process} in an operating system (OS), but with much
less overhead.  Threaded applications have become quite popular in even
uniprocessor systems, and Unix,\footnote{Here and below, the term {\it
Unix} includes Linux.} Windows, Python, Java, Perl and now C++11 and R
(via my Rdsm package) all support threaded programming.

In the typical implementation, a thread is a special case of an OS
process.  But the key difference is that the various threads of a
program share memory.  (One can arrange for processes to share memory
too in some OSs, but they don't do so by default.)

On a uniprocessor system, the threads of a program take turns executing,
so that there is only an illusion of parallelism.  But on a
multiprocessor system, one can genuinely have threads running in
parallel.\footnote{There may be other processes running too.  So the
threads of our program must still take turns with other processes
running on the machine.}  Whenever a processor becomes available, the OS
will assign some ready thread to it.  So, among other things, this says
that a thread might actually run on different processors during
different turns.

\label{needknowos}
{\bf Important note:}  Effective use of threads requires a basic
understanding of how processes take turns executing.  See Section
\ref{processes} in the appendix of this book for this material.

One of the most popular threads systems is Pthreads, whose name is short
for POSIX threads.  POSIX is a Unix standard, and the Pthreads system
was designed to standardize threads programming on Unix.  It has since
been ported to other platforms.

\subsubsection{Example: Pthreads Prime Numbers Finder}
\label{pthreadsprime}

Following is an example of Pthreads programming, in which we determine
the number of prime numbers in a certain range.  Read the comments at
the top of the file for details; the threads operations will be
explained presently.

\begin{Verbatim}
// PrimesThreads.c

// threads-based program to find the number of primes between 2 and n;
// uses the Sieve of Eratosthenes, deleting all multiples of 2, all
// multiples of 3, all multiples of 5, etc.

// for illustration purposes only; NOT claimed to be efficient

// Unix compilation:  gcc -g -o primesthreads PrimesThreads.c -lpthread -lm

// usage:  primesthreads n num_threads

#include <stdio.h>
#include <math.h>
#include <pthread.h>  // required for threads usage

#define MAX_N 100000000
#define MAX_THREADS 25

// shared variables
int nthreads,  // number of threads (not counting main())
    n,  // range to check for primeness
    prime[MAX_N+1],  // in the end, prime[i] = 1 if i prime, else 0
    nextbase;  // next sieve multiplier to be used
// lock for the shared variable nextbase
pthread_mutex_t nextbaselock = PTHREAD_MUTEX_INITIALIZER;
// ID structs for the threads
pthread_t id[MAX_THREADS];

// "crosses out" all odd multiples of k
void crossout(int k)
{  int i;
   for (i = 3; i*k <= n; i += 2)  {
      prime[i*k] = 0;
   }
}

// each thread runs this routine
void *worker(int tn)  // tn is the thread number (0,1,...)
{  int lim,base,
       work = 0;  // amount of work done by this thread
   // no need to check multipliers bigger than sqrt(n)
   lim = sqrt(n);
   do  {
      // get next sieve multiplier, avoiding duplication across threads
      // lock the lock
      pthread_mutex_lock(&nextbaselock);
      base = nextbase;
      nextbase += 2;
      // unlock
      pthread_mutex_unlock(&nextbaselock);
      if (base <= lim)  {
         // don't bother crossing out if base known composite
         if (prime[base])  {
            crossout(base);
            work++;  // log work done by this thread
         }
      }
      else return work;
   } while (1);
}

main(int argc, char **argv)
{  int nprimes,  // number of primes found
       i,work;
   n = atoi(argv[1]);
   nthreads = atoi(argv[2]);
   // mark all even numbers nonprime, and the rest "prime until
   // shown otherwise"
   for (i = 3; i <= n; i++)  {
      if (i%2 == 0) prime[i] = 0;
      else prime[i] = 1;
   }
   nextbase = 3;
   // get threads started
   for (i = 0; i < nthreads; i++)  {
      // this call says create a thread, record its ID in the array
      // id, and get the thread started executing the function worker(),
      // passing the argument i to that function
      pthread_create(&id[i],NULL,worker,i);
   }

   // wait for all done
   for (i = 0; i < nthreads; i++)  {
      // this call says wait until thread number id[i] finishes
      // execution, and to assign the return value of that thread to our
      // local variable work here
      pthread_join(id[i],&work);
      printf("%d values of base done\n",work);
   }

   // report results
   nprimes = 1;
   for (i = 3; i <= n; i++)
      if (prime[i])  {
         nprimes++;
      }
   printf("the number of primes found was %d\n",nprimes);

}
\end{Verbatim}

To make our discussion concrete, suppose we are running this program
with two threads.  Suppose also the both threads are running
simultaneously most of the time.  This will occur if they aren't
competing for turns with other threads, say if there are no other
threads, or more generally if the number of other threads is
less than or equal to the number of processors minus two.  (Actually,
the original thread is {\bf main()}, but it lies dormant most of the
time, as you'll see.)

Note the global variables:

\begin{Verbatim}
int nthreads,  // number of threads (not counting main())
    n,  // range to check for primeness
    prime[MAX_N+1],  // in the end, prime[i] = 1 if i prime, else 0
    nextbase;  // next sieve multiplier to be used
pthread_mutex_t nextbaselock = PTHREAD_MUTEX_INITIALIZER;
pthread_t id[MAX_THREADS];
\end{Verbatim}

This will require some adjustment for those who've been taught that
global variables are ``evil.''

In most threaded programs, all communication between threads is done via
global variables.\footnote{Technically one could use locals in {\bf
main()} (or whatever function it is where the threads are created) for
this purpose, but this would be so unwieldy that it is seldom done.} So
even if you consider globals to be evil, they are a necessary evil in
threads programming.

Personally I have always thought the stern admonitions against global
variables are overblown anyway; see
\url{http://heather.cs.ucdavis.edu/~matloff/globals.html}.  But as
mentioned, those admonitions are routinely ignored in threaded
programming.  For a nice discussion on this, see the paper by a famous
MIT computer scientist on an Intel Web page, at
\url{http://software.intel.com/en-us/articles/global-variable-reconsidered/?wapkw=%28parallelism%29}.

As mentioned earlier, the globals are shared by all
processors.\footnote{Technically, we should say ``shared by all
threads'' here, as a given thread does not always execute on the same
processor, but at any instant in time each executing thread is at some
processor, so the statement is all right.} If one processor, for
instance, assigns the value 0 to {\bf prime{[}35{]}} in the function
{\bf crossout()}, then that variable will have the value 0 when accessed
by any of the other processors as well.  On the other hand, local
variables have different values at each processor; for instance, the
variable {\bf i} in that function has a different value at each
processor.

Note that in the statement

\begin{Verbatim}
pthread_mutex_t nextbaselock = PTHREAD_MUTEX_INITIALIZER;
\end{Verbatim}

the right-hand side is not a constant.  It is a macro call, and is thus
something which is executed.

In the code

\begin{Verbatim}
pthread_mutex_lock(&nextbaselock);
base = nextbase
nextbase += 2
pthread_mutex_unlock(&nextbaselock);
\end{Verbatim}

we see a \textbf{critical section} operation which is
typical in shared-memory programming.  In this context here, it means
that we cannot allow more than one thread to execute the code

\begin{Verbatim}
base = nextbase;
nextbase += 2;
\end{Verbatim}

at the same time.  A common term used for this is that we wish the
actions in the critical section to collectively be {\bf atomic}, meaning
not divisible among threads.  The calls to {\bf pthread\_mutex\_lock()}
and {\bf pthread\_mutex\_unlock()} ensure this.  If thread A is
currently executing inside the critical section and thread B tries to
lock the lock by calling {\bf pthread\_mutex\_lock()}, the call will
block until thread B executes {\bf pthread\_mutex\_unlock()}.

Here is why this is so important:  Say currently {\bf nextbase} has the
value 11.  What we want to happen is that the next thread to read {\bf
nextbase} will ``cross out'' all multiples of 11.  But if we allow
two threads to execute the critical section at the same time, the
following may occur, in order:

\begin{itemize}

\item thread A reads {\bf nextbase}, setting its value of {\bf base} to
11

\item thread B reads {\bf nextbase}, setting its value of {\bf base} to
11

\item thread A adds 2 to {\bf nextbase}, so that {\bf nextbase} becomes
13

\item thread B adds 2 to {\bf nextbase}, so that {\bf nextbase} becomes
15

\end{itemize}

Two problems would then occur:

\begin{itemize}

\item Both threads would do ``crossing out'' of multiples of 11,
duplicating work and thus slowing down execution speed.

\item We will never ``cross out'' multiples of 13.

\end{itemize}

Thus the lock is crucial to the correct (and speedy) execution of the
program.

Note that these problems could occur either on a uniprocessor or
multiprocessor system.  In the uniprocessor case, thread A's turn might
end right after it reads {\bf nextbase}, followed by a turn by B which
executes that same instruction.  In the multiprocessor case, A and B
could literally be running simultaneously, but still with the action by
B coming an instant after A.

This problem frequently arises in parallel database systems.  For
instance, consider an airline reservation system. If a flight has only
one seat left, we want to avoid giving it to two different customers who
might be talking to two agents at the same time. The lines of code in
which the seat is finally assigned (the \textbf{commit} phase, in
database terminology) is then a critical section.

A critical section is always a potential bottleneck in a parallel
program, because its code is serial instead of parallel.  In our program
here, we may get better performance by having each thread work on, say,
five values of {\bf nextbase} at a time.  Our line

\begin{Verbatim}
nextbase += 2;
\end{Verbatim}

would become

\begin{Verbatim}
nextbase += 10;
\end{Verbatim}

That would mean that any given thread would need to go through the
critical section only one-fifth as often, thus greatly reducing
overhead.  On the other hand, near the end of the run, this may result
in some threads being idle while other threads still have a lot of work
to do.


Note this code.

\begin{Verbatim}
for (i = 0; i < nthreads; i++)  {
   pthread_join(id[i],&work);
   printf("%d values of base done\n",work);
}
\end{Verbatim}

This is a special case of of {\bf barrier}.

A barrier is a point in the code that all threads must reach before
continuing.  In this case, a barrier is needed in order to prevent
premature execution of the later code

\begin{Verbatim}
for (i = 3; i <= n; i++)
   if (prime[i])  {
      nprimes++;
   }
\end{Verbatim}

which would result in possibly wrong output if we start counting primes
before some threads are done.

Actually, we could have used Pthreads' built-in barrier function.  We
need to declare a barrier variable, e.g.

\begin{lstlisting}
pthread_barrier_t barr;
\end{lstlisting}

and then call it like this:

\begin{lstlisting}
pthread_barrier_wait(&barr);
\end{lstlisting}

The {\bf pthread\_join()} function actually causes the given thread to
exit, so that we then ``join'' the thread that created it, i.e. {\bf
main()}.  Thus some may argue that this is not really a true barrier.

Barriers are very common in shared-memory programming, and will be
discussed in more detail in Chapter \ref{chap:shared}.

\subsubsection{操作系统的角色}

Let's again ponder the role of the OS here.  What happens when a thread
tries to lock a lock:

\begin{itemize}

\item The lock call will ultimately cause a system call, causing the OS
to run.

\item The OS keeps track of the locked/unlocked status of each lock, so
it will check that status.

\item Say the lock is unlocked (a 0).  Then  the OS sets it to locked (a
1), and the lock call returns.  The thread enters the critical section.

\item When the thread is done, the unlock call unlocks the lock, similar
to the locking actions.

\item If the lock is locked at the time a thread makes a lock call, the
call will block.  The OS will mark this thread as waiting for the lock.
When whatever thread currently using the critical section unlocks the
lock, the OS will relock it and unblock the lock call of the waiting
thread.  If several threads are waiting, of course only one will be
unblock.

\end{itemize}

Note that {\bf main()} is a thread too, the original thread that spawns
the others.  However, it is dormant most of the time, due to its calls
to {\bf pthread\_join()}.

Finally, keep in mind that although the globals variables are shared,
the locals are not.  Recall that local variables are stored on a stack.
Each thread (just like each process in general) has its own stack.  When
a thread begins a turn, the OS prepares for this by pointing the stack
pointer register to this thread's stack.

\subsubsection{调试 Threads 程序}
\label{debugthreads}

Most debugging tools include facilities for threads.  Here's an overview
of how it works in GDB.

First, as you run a program under GDB, the creation of new threads will
be announced, e.g.

\begin{Verbatim}
(gdb) r 100 2
Starting program: /debug/primes 100 2
[New Thread 16384 (LWP 28653)]
[New Thread 32769 (LWP 28676)]
[New Thread 16386 (LWP 28677)]
[New Thread 32771 (LWP 28678)]
\end{Verbatim}

You can do backtrace ({\bf bt}) etc. as usual.  Here are some
threads-related commands:

\begin{itemize}

\item {\tt info threads} (gives information on all current threads)

\item {\tt thread 3} (change to thread 3)

\item {\tt break 88 thread 3} (stop execution when thread 3 reaches
source line 88)

\item {\tt break 88 thread 3 if x==y} (stop execution when thread 3 reaches
source line 88 and the variables x and y are equal)

\end{itemize}

Of course, many GUI IDEs use GDB internally, and thus provide the above
facilities with a GUI wrapper.  Examples are DDD, Eclipse and NetBeans.

\subsubsection{Higher-Level Threads}

The OpenMP library gives the programmer a higher-level view of
threading.  The threads are there, but rather hidden by higher-level
abstractions.  We will study OpenMP in detail in Chapter \ref{chap:omp},
and use it frequently in the succeeding chapters, but below is an
introductory example.

\subsubsection{示例:Sampling Bucket Sort}
\label{ompbsort}

This code implements the sampling bucket sort of Section \ref{bsort}.

\begin{lstlisting}
// OpenMP introductory example:  sampling bucket sort

// compile:  gcc -fopenmp -o bsort bucketsort.c

// set the number of threads via the environment variable
// OMP_NUM_THREADS, e.g. in the C shell

// setenv OMP_NUM_THREADS 8

#include <omp.h>  // required
#include <stdlib.h>

// needed for call to qsort()
int cmpints(int *u, int *v)
{  if (*u < *v) return -1;
   if (*u > *v) return 1;
   return 0;
}

// adds xi to the part array, increments npart, the length of part
void grab(int xi, int *part, int *npart)
{
    part[*npart] = xi;
    *npart += 1;
}

// finds the min and max in y, length ny,
// placing them in miny and maxy
void findminmax(int *y, int ny, int *miny, int *maxy)
{  int i,yi;
   *miny = *maxy = y[0];
   for (i = 1; i < ny; i++) {
      yi = y[i];
      if (yi < *miny) *miny = yi;
      else if (yi > *maxy) *maxy = yi;
   }
}

// sort the array x of length n
void bsort(int *x, int n)
{  // these are local to this function, but shared among the threads
   float *bdries; int *counts;
   #pragma omp parallel
   // entering this block activates the threads, each executing it
   {  // variables declared below are local to each thread
      int me = omp_get_thread_num();
      // have to do the next call within the block, while the threads
      // are active
      int nth = omp_get_num_threads();
      int i,xi,minx,maxx,start;
      int *mypart;
      float increm;
      int SAMPLESIZE;
      // now determine the bucket boundaries; nth - 1 of them, by
      // sampling the array to get an idea of its range
      #pragma omp single  // only 1 thread does this, implied barrier at end
      {
         if (n > 1000) SAMPLESIZE = 1000;
         else SAMPLESIZE = n / 2;
         findminmax(x,SAMPLESIZE,&minx,&maxx);
         bdries = malloc((nth-1)*sizeof(float));
         increm = (maxx - minx) / (float) nth;
         for (i = 0; i < nth-1; i++)
            bdries[i] = minx + (i+1) * increm;
         // array to serve as the count of the numbers of elements of x
         // in each bucket
         counts = malloc(nth*sizeof(int));
      }
      // now have this thread grab its portion of the array; thread 0
      // takes everything below bdries[0], thread 1 everything between
      // bdries[0] and bdries[1], etc., with thread nth-1 taking
      // everything over bdries[nth-1]
      mypart = malloc(n*sizeof(int)); int nummypart = 0;
      for (i = 0; i < n; i++) {
         if (me == 0) {
            if (x[i] <= bdries[0]) grab(x[i],mypart,&nummypart);
         }
         else if (me < nth-1) {
            if (x[i] > bdries[me-1] && x[i] <= bdries[me])
               grab(x[i],mypart,&nummypart);
         } else
            if (x[i] > bdries[me-1]) grab(x[i],mypart,&nummypart);
      }
      // now record how many this thread got
      counts[me] = nummypart;
      // sort my part
      qsort(mypart,nummypart,sizeof(int),cmpints);
      #pragma omp barrier  // other threads need to know all of counts
      // copy sorted chunk back to the original array; first find start point
      start = 0;
      for (i = 0; i < me; i++) start += counts[i];
      for (i = 0; i < nummypart; i++) {
         x[start+i] = mypart[i];
      }
   }
   // implied barrier here; main thread won't resume until all threads
   // are done
}

int main(int argc, char **argv)
{
   // test case
   int n = atoi(argv[1]), *x = malloc(n*sizeof(int));
   int i;
   for (i = 0; i < n; i++) x[i] = rand() % 50;
   if (n < 100)
      for (i = 0; i < n; i++) printf("%d\n",x[i]);
   bsort(x,n);
   if (n <= 100) {
      printf("x after sorting:\n");
      for (i = 0; i < n; i++) printf("%d\n",x[i]);
   }
}
\end{lstlisting}

Details on OpenMP are presented in Chapter \ref{chap:omp}.  Here is an
overview of a few of the OpenMP constructs available:

\begin{itemize}

\item \textbf{\#pragma omp for}

In our example above, we wrote our own code to assign specific threads to
do specific parts of the work.  An alternative is to write an ordinary
{\bf for} loop that iterates over all the work to be done, and then ask
OpenMP to assign specific iterations to specific threads.  To do this,
insert the above pragma just before the loop.

\item \textbf{\#pragma omp critical}

The block that follows is implemented as a critical section.  OpenMP
sets up the locks etc. for you, alleviating you of work and alleviating
your code of clutter.

\end{itemize}

\subsubsection{调试 OpenMP}

Since there are threads underlying the OpenMP execution, you should be
able to use your debugging tool's threads facilities.  Note, though,
that this may not work perfectly well.

Some versions of GCC/GDB, for instance, do not display some local
variables.  Let's consider two categories of such variables:

\begin{itemize}

\item [(a)] Variables within a {\bf parallel} block, such as {\bf me}
in {\bf bsort()} in Section \ref{ompbsort}.

\item [(b)] Variables that are not in a {\bf parallel} block, but which
are still local to a function that contains such a block.  An example is
{\bf counts} in {\bf bsort()}.

\end{itemize}

You may find that when you try to use GDB's {\bf print} command, GDB
says there is no such variable.

The problem seems to arise from a combination of (i) optimzation, so
that a variable is placed in a register and basically eliminated from
the namespace, and (ii) some compilers implement OpenMP by actually
making special versions of the function being debugged.

In GDB, one possible workaround is to use the {\bf -gstabs+} option when
compiling, instead of {\bf -g}.  But here is a more general workarounds.
Let's consider variables of type (b) first.

The solution is to temporarily change these variables to globals, e.g.

\begin{lstlisting}
int *counts;
void bsort(int *x, int n)
\end{lstlisting}

This would still be all right in terms of program correctness, because
the variables in (b) are global to the threads anyway. (Of course, make
sure not to have another global of the same name!) The switch would only
be temporary, during debugging, to be switched back later so that in the
end {\bf bsort()} is self-contained.

The same solution works for category (a) variables, with an added line:

\begin{lstlisting}
int me;
#pragma omp threadprivate(me)
void bsort(int *x, int n)
\end{lstlisting}

What this does is make separate copies of {\bf me} as global variables, one
for each thread. As globals, GCC won't engage in any shenanigans with
them. :-) One does have to keep in mind that they will retain there
values upon exit from a parallel block etc., but the workaround does
work.



\subsection{Message Passing}

\subsubsection{Programmer View}
\label{msgview}

Again consider the matrix-vector multiply example of Section
\ref{matvec}.  In contrast to the shared-memory case, in the
message-passing paradigm all nodes would have \underbar{separate} copies
of A, X and Y.  Our example in Section \ref{sharedview} would now
change.  in order for node 2 to send this new value of Y{[}3{]} to node
15, it would have to execute some special function, which would be
something like

\begin{Verbatim}
   send(15,12,"Y[3]");
\end{Verbatim}

and node 15 would have to execute some kind of {\bf receive()} function.

To compute the matrix-vector product, then, would involve the following.
One node, say node 0, would distribute the rows of A to the various
other nodes.  Each node would receive a different set of rows.  The
vector X would be sent to all nodes.\footnote{In a more refined version,
X would be parceled out to the nodes, just as the rows of A are.} Each
node would then multiply X by the node's assigned rows of A, and then
send the result back to node 0.  The latter would collect those results,
and store them in Y.

\subsubsection{Example:  MPI Prime Numbers Finder}
\label{mpiex}

Here we use the MPI system, with our hardware being a cluster.

MPI is a popular public-domain set of interface functions, callable from
C/C++, to do message passing.  We are again counting primes, though in
this case using a \textbf{pipelining} method.  It is similar to hardware
pipelines, but in this case it is done in software, and each ``stage''
in the pipe is a different computer.

The program is self-documenting, via the comments.

\begin{Verbatim}
/* MPI sample program; NOT INTENDED TO BE EFFICIENT as a prime
   finder, either in algorithm or implementation

   MPI (Message Passing Interface) is a popular package using
   the "message passing" paradigm for communicating between
   processors in parallel applications; as the name implies,
   processors communicate by passing messages using "send" and
   "receive" functions

   finds and reports the number of primes less than or equal to N

   uses a pipeline approach:  node 0 looks at all the odd numbers (i.e.
   has already done filtering out of multiples of 2) and filters out
   those that are multiples of 3, passing the rest to node 1; node 1
   filters out the multiples of 5, passing the rest to node 2; node 2
   then removes the multiples of 7, and so on; the last node must check
   whatever is left

   note that we should NOT have a node run through all numbers
   before passing them on to the next node, since we would then
   have no parallelism at all; on the other hand, passing on just
   one number at a time isn't efficient either, due to the high
   overhead of sending a message if it is a network (tens of
   microseconds until the first bit reaches the wire, due to
   software delay); thus efficiency would be greatly improved if
   each node saved up a chunk of numbers before passing them to
   the next node */

#include <mpi.h>  // mandatory

#define PIPE_MSG 0  // type of message containing a number to be checked
#define END_MSG 1  // type of message indicating no more data will be coming

int NNodes,  // number of nodes in computation
    N,  // find all primes from 2 to N
    Me;  // my node number
double T1,T2;  // start and finish times

void Init(int Argc,char **Argv)
{  int DebugWait;
   N = atoi(Argv[1]);
   // start debugging section
   DebugWait = atoi(Argv[2]);
   while (DebugWait) ;  // deliberate infinite loop; see below
   /* the above loop is here to synchronize all nodes for debugging;
      if DebugWait is specified as 1 on the mpirun command line, all
      nodes wait here until the debugging programmer starts GDB at
      all nodes (via attaching to OS process number), then sets
      some breakpoints, then GDB sets DebugWait to 0 to proceed; */
   // end debugging section
   MPI_Init(&Argc,&Argv);  // mandatory to begin any MPI program
   // puts the number of nodes in NNodes
   MPI_Comm_size(MPI_COMM_WORLD,&NNodes);
   // puts the node number of this node in Me
   MPI_Comm_rank(MPI_COMM_WORLD,&Me);
   // OK, get started; first record current time in T1
   if (Me == NNodes-1) T1 = MPI_Wtime();
}

void Node0()
{  int I,ToCheck,Dummy,Error;
   for (I = 1; I <= N/2; I++)  {
      ToCheck = 2 * I + 1;  // latest number to check for div3
      if (ToCheck > N) break;
      if (ToCheck % 3 > 0)  // not divis by 3, so send it down the pipe
         // send the string at ToCheck, consisting of 1 MPI integer, to
         // node 1 among MPI_COMM_WORLD, with a message type PIPE_MSG
         Error = MPI_Send(&ToCheck,1,MPI_INT,1,PIPE_MSG,MPI_COMM_WORLD);
         // error not checked in this code
   }
   // sentinel
   MPI_Send(&Dummy,1,MPI_INT,1,END_MSG,MPI_COMM_WORLD);
}

void NodeBetween()
{  int ToCheck,Dummy,Divisor;
   MPI_Status Status;
   // first received item gives us our prime divisor
   // receive into Divisor 1 MPI integer from node Me-1, of any message
   // type, and put information about the message in Status
   MPI_Recv(&Divisor,1,MPI_INT,Me-1,MPI_ANY_TAG,MPI_COMM_WORLD,&Status);
   while (1)  {
      MPI_Recv(&ToCheck,1,MPI_INT,Me-1,MPI_ANY_TAG,MPI_COMM_WORLD,&Status);
      // if the message type was END_MSG, end loop
      if (Status.MPI_TAG == END_MSG) break;
      if (ToCheck % Divisor > 0)
         MPI_Send(&ToCheck,1,MPI_INT,Me+1,PIPE_MSG,MPI_COMM_WORLD);
   }
   MPI_Send(&Dummy,1,MPI_INT,Me+1,END_MSG,MPI_COMM_WORLD);
}

NodeEnd()
{  int ToCheck,PrimeCount,I,IsComposite,StartDivisor;
   MPI_Status Status;
   MPI_Recv(&StartDivisor,1,MPI_INT,Me-1,MPI_ANY_TAG,MPI_COMM_WORLD,&Status);
   PrimeCount = Me + 2;  /* must account for the previous primes, which
                            won't be detected below */
   while (1)  {
      MPI_Recv(&ToCheck,1,MPI_INT,Me-1,MPI_ANY_TAG,MPI_COMM_WORLD,&Status);
      if (Status.MPI_TAG == END_MSG) break;
      IsComposite = 0;
      for (I = StartDivisor; I*I <= ToCheck; I += 2)
         if (ToCheck % I == 0)  {
            IsComposite = 1;
            break;
         }
      if (!IsComposite) PrimeCount++;
   }
   /* check the time again, and subtract to find run time */
   T2 = MPI_Wtime();
   printf("elapsed time = %f\n",(float)(T2-T1));
   /* print results */
   printf("number of primes = %d\n",PrimeCount);
}

int main(int argc,char **argv)
{  Init(argc,argv);
   // all nodes run this same program, but different nodes take
   // different actions
   if (Me == 0) Node0();
   else if (Me == NNodes-1) NodeEnd();
        else NodeBetween();
   // mandatory for all MPI programs
   MPI_Finalize();
}

/* explanation of "number of items" and "status" arguments at the end
   of MPI_Recv():

   when receiving a message you must anticipate the longest possible
   message, but the actual received message may be much shorter than
   this; you can call the MPI_Get_count() function on the status
   argument to find out how many items were actually received

   the status argument will be a pointer to a struct, containing the
   node number, message type and error status of the received
   message

   say our last parameter is Status; then Status.MPI_SOURCE
   will contain the number of the sending node, and
   Status.MPI_TAG will contain the message type; these are
   important if used MPI_ANY_SOURCE or MPI_ANY_TAG in our
   node or tag fields but still have to know who sent the
   message or what kind it is */
\end{Verbatim}

The set of machines can be heterogeneous, but MPI ``translates'' for you
automatically.  If say one node has a big-endian CPU and another has a
little-endian CPU, MPI will do the proper conversion.

\subsection{Scatter/Gather}

Technically, the {\bf scatter/gather} programmer world view is a special
case of message passing.  However, it has become so pervasive as to
merit its own section here.

In this paradigm, one node, say node 0, serves as a {\bf manager}, while
the others serve as {\bf workers}.  The parcels out work to the workers,
who process their respective chunks of the data and return the results
to the manager.  The latter receives the results and combines them into
the final product.

The matrix-vector multiply example in Section \ref{msgview} is an
example of scatter/gather.

As noted, scatter/gather is very popular.  Here are some examples of
packages that use it:

\begin{itemize}

\item MPI includes scatter and gather functions (Section
\ref{scattergather}).

\item Hadoop/MapReduce Computing (Chapter \ref{chap:cloud}) is basically a
scatter/gather operation.

\item The {\bf snow} package (Section \ref{snowintro}) for
the  R language is also a scatter/gather operation.

\end{itemize}

\subsubsection{R snow Package}
\label{snowintro}

Base R does not include parallel processing facilities, but includes the
{\bf parallel} library for this purpose, and a number of other parallel
libraries are available as well.  The {\bf parallel} package arose from
the merger (and slight modifcation) of two former user-contributed
libraries, {\bf snow} and {\bf multicore}.  The former (and essentially
the latter) uses the scatter/gather paradigm, and so will be introduced
in this section; see Section \ref{snowintro} for further details.  for
convenience, I'll refer to the portion of {\bf parallel} that came from
{\bf snow} simply as {\bf snow}.

Let's use matrix-vector multiply as an example to learn from:

\begin{lstlisting}
> library(parallel)
> c2 <- makePSOCKcluster(rep("localhost",2))
> c2
socket cluster with 2 nodes on host ‘localhost’
> mmul
function(cls,u,v) {
   rowgrps <- splitIndices(nrow(u),length(cls))
   grpmul <- function(grp) u[grp,] %*% v
   mout <- clusterApply(cls,rowgrps,grpmul)
   Reduce(c,mout)
}
> a <- matrix(sample(1:50,16,replace=T),ncol=2)
> a
     [,1] [,2]
[1,]   34   41
[2,]   10   28
[3,]   44   23
[4,]    7   29
[5,]    6   24
[6,]   28   29
[7,]   21    1
[8,]   38   30
> b <- c(5,-2)
> b
[1]  5 -2
> a %*% b  # serial multiply
     [,1]
[1,]   88
[2,]   -6
[3,]  174
[4,]  -23
[5,]  -18
[6,]   82
[7,]  103
[8,]  130
> clusterExport(c2,c("a","b"))  # send a,b to workers
> clusterEvalQ(c2,a)  # check that they have it
[[1]]
     [,1] [,2]
[1,]   34   41
[2,]   10   28
[3,]   44   23
[4,]    7   29
[5,]    6   24
[6,]   28   29
[7,]   21    1
[8,]   38   30

[[2]]
     [,1] [,2]
[1,]   34   41
[2,]   10   28
[3,]   44   23
[4,]    7   29
[5,]    6   24
[6,]   28   29
[7,]   21    1
[8,]   38   30
> mmul(c2,a,b)  # test our parallel code
[1]  88  -6 174 -23 -18  82 103 130
\end{lstlisting}

What just happened?

First we set up a {\bf snow} cluster.  The term should not be confused
with hardware systems we referred to as ``clusters'' earlier.  We are
simply setting up a group of R processes that will communicate with each
other via TCP/IP sockets.

In this case, my cluster consists of two R processes running on the
machine from which I invoked {\bf makePSOCKcluster()}.  (In TCP/IP
terminology, {\bf localhost} refers to the local machine.)  If I were to
run the Unix {\bf ps} command, with appropriate options, say {\bf ax},
I'd see three R processes.  I saved the cluster in {\bf c2}.

On the other hand, my {\bf snow} cluster could indeed be set up on a
real cluster, e.g.

\begin{lstlisting}
c3 <- makePSOCKcluster(c("pc28","pc29","pc29"))
\end{lstlisting}

where {\bf pc28} etc. are machine names.

In preparing to test my parallel code, I needed to ship my matrices {\bf
a} and {\bf b} to the workers:

\begin{lstlisting}
> clusterExport(c2,c("a","b"))  # send a,b to workers
\end{lstlisting}

Note that this function assumes that {\bf a} and {\bf b} are global
variables at the invoking node, i.e. the manager, and it will place
copies of them in the global workspace of the worker nodes.

Note that the copies are independent of the originals; if a worker
changes, say, {\bf b[3]}, that change won't be made at the manager or at
the other worker.  This is a message-passing system, indeed.

So, how does the {\bf mmul} code work?  Here's a handy copy:

\begin{lstlisting}
mmul <- function(cls,u,v) {
   rowgrps <- splitIndices(nrow(u),length(cls))
   grpmul <- function(grp) u[grp,] %*% v
   mout <- clusterApply(cls,rowgrps,grpmul)
   Reduce(c,mout)
}
\end{lstlisting}

As discussed in Section \ref{matvec}, our strategy will be to partition
the rows of the matrix, and then have different workers handle different
groups of rows.  Our call to {\bf splitIndices()} sets this up for us.

That function does what its name implies, e.g.

\begin{lstlisting}
> splitIndices(12,5)
[[1]]
[1] 1 2 3

[[2]]
[1] 4 5

[[3]]
[1] 6 7

[[4]]
[1] 8 9

[[5]]
[1] 10 11 12
\end{lstlisting}

Here we asked the function to partition the numbers 1,...,12 into 5
groups, as equal-sized as possible, which you can see is what it did.
Note that the type of the return value is an R list.

So, after executing that function in our {\bf mmul()} code, {\bf
rowgrps} will be an R list consisting of a partitioning of the row
numbers of {\bf u}, exactly what we need.

The call to {\bf clusterApply()} is then where the actual work is
assigned to the workers.  The code

\begin{lstlisting}
mout <- clusterApply(cls,rowgrps,grpmul)
\end{lstlisting}

instructs {\bf snow} to have the first worker process the rows in {\bf
rowgrps[[1]]}, the second worker to work on {\bf rowgrps[[2]]},
and so on.  The {\bf clusterApply()} function expects its second
argument to be an R list, which is the case here.

Each worker will then multiply {\bf v} by its row group, and
return the product to the manager.  However, the product will again be a
list, one component for each worker, so we need {\bf Reduce()} to string
everything back together.

Note that R does allow functions defined within functions, which the
locals and arguments of the outer function becoming global to the inner
function.

Note that {\bf a} here could have been huge, in which case the export
action could slow down our program.  If {\bf a} were not needed at the
workers other than for this one-time matrix multiply, we may wish to
change to code so that we send each worker only the rows of {\bf a} that
we need:

\begin{lstlisting}
mmul1 <- function(cls,u,v) {
   rowgrps <- splitIndices(nrow(u),length(cls))
   uchunks <- Map(function(grp) u[grp,],rowgrps)
   mulchunk <- function(uc) uc %*% v
   mout <- clusterApply(cls,uchunks,mulchunk)
   Reduce(c,mout)
}
\end{lstlisting}

Let's test it:

\begin{lstlisting}
> a <- matrix(sample(1:50,16,replace=T),ncol=2)
> b <- c(5,-2)
> clusterExport(c2,"b")  # don't send a
 a
     [,1] [,2]
[1,]   10   26
[2,]    1   34
[3,]   49   30
[4,]   39   41
[5,]   12   14
[6,]    2   30
[7,]   33   23
[8,]   44    5
> a %*% b
     [,1]
[1,]   -2
[2,]  -63
[3,]  185
[4,]  113
[5,]   32
[6,]  -50
[7,]  119
[8,]  210
> mmul1(c2,a,b)
[1]  -2 -63 185 113  32 -50 119 210
\end{lstlisting}

Note that we did not need to use {\bf clusterExport()} to send the
chunks of {\bf a} to the workers, as the call to {\bf clusterApply()}
does this, since it sends the arguments,
